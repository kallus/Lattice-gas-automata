\documentclass[12pt,a4paper]{article}
\usepackage{amssymb}
%\usepackage{latexsym}
\usepackage{amsmath}
\usepackage{listings}
\usepackage{graphicx}
\usepackage{color}
\usepackage{url}
\usepackage{caption}
\usepackage{subfigure}

\captionsetup{width=.8\textwidth,font=small} 

% Stolen settings (unknown origin):
% Alter some LaTeX defaults for better treatment of figures:
% See p.105 of "TeX Unbound" for suggested values.
% See pp. 199-200 of Lamport's "LaTeX" book for details.
%   General parameters, for ALL pages:
\renewcommand{\topfraction}{0.9}	% max fraction of floats at top
\renewcommand{\bottomfraction}{0.8}	% max fraction of floats at bottom
%   Parameters for TEXT pages (not float pages):
\setcounter{topnumber}{2}
\setcounter{bottomnumber}{2}
\setcounter{totalnumber}{4}     % 2 may work better
\setcounter{dbltopnumber}{2}    % for 2-column pages
\renewcommand{\dbltopfraction}{0.9}	% fit big float above 2-col. text
\renewcommand{\textfraction}{0.07}	% allow minimal text w. figs
%   Parameters for FLOAT pages (not text pages):
\renewcommand{\floatpagefraction}{0.7}	% require fuller float pages
% N.B.: floatpagefraction MUST be less than topfraction !!
\renewcommand{\dblfloatpagefraction}{0.7}	% require fuller float pages

% remember to use [htp] or [htpb] for placement

\author{Rasmus Einarsson, Jonatan Kallus, Olof Nilsson, Joel Wilsson}
\title{Simulation of Complex Systems\\Lattice gas automata}

\begin{document}
\maketitle

\section{Introduction}
Lattice gas automata are a kind of cellular automata, which are used to model some aspects of the physics of gases, such as diffusion and waves.

This text explores the behaviour of lattice gas automata. A number of simulations are run to qualitatively see if the model captures some phenomena expected from actual gases. The investigated phenomena are specifically waves, diffusion and currents of a flow around a cylinder. Some attempts are made to make meaningful comparisons to the theory of gas physics.


The theory of lattice gas automata in general is introduced in section~\ref{sec:theory} followed by the specific details of our implementation in section~\ref{sec:imp}. The conducted experiments and their results are presented in section~\ref{sec:exp}. In the last section, section~\ref{sec:conclusion}, we conclude that the model can capture waves and diffusion, but not the typical behaviour of a compressed flow around a cylinder. There we also discuss the benefits and disadvantages of the model, and that the model can show a turbulent behaviour under certain circumstances.

Cellular automata in general, thus also lattice gas automata, are systems that are discrete in time and in space. They are built using a small set of deterministic rules, which make up a mapping from each possible spatial configuration of some small area into a spatial configuration of that area for the next time step.

The major benefit of simulation of gases using lattice gas automata is simplicity and speed, as the model is built using only simple rules. Scientific interest in lattice gas automata declined drastically in the beginning of the 1990's, probably due to the increased interest in more sophisticated models using Lattice Boltzmann methods. The new models made it possible to capture more of the phenomena of fluid physics, and were made possible by faster computers.

The source code for this project has been made public at \url{https://github.com/kallus/Lattice-gas-automata}.

\section{Theory}
\label{sec:theory}
The lattice gas model describes how particles move on a lattice. Particles make discrete jumps between lattice nodes, where they may collide with each other and change directions. In the following discussion, we only consider two-dimensional lattices.

Specifically, the model consists of a lattice where each node takes on one of a finite set of states. Each state represents a configuration of particles in that site (we will use ``site'' and ``node'' interchangeably). In each time step, the nodes are updated according to a rule based on its value and the values of the nearest neighbors. The update rule corresponds to particle collisions within nodes and movements between neighboring nodes.

\subsection{Square lattice example}
Consider the example of a square lattice, where each node has four neighbors. For a particle in a node, there are four possible velocities (up, down, left, right). For each lattice node, there can be at most one particle moving in each direction. Hence, the state of a node can be encoded in four binary values indicating whether there is or is not a particle moving in each direction. See figure \ref{fig:square-state} for an illustration.

\begin{figure}[htp]
\centering
  \includegraphics{figs/square-state.pdf}
\caption{Example showing the state of four nodes. Each node can be occupied by zero to four particles, one in each direction. The state $n_{ij}$ can be encoded using four binary values \{up, right, down, left\} as indicated the figure. For instance, in the upper right corner, there is one particle moving to the left. In the lower right corner, there are particles moving upwards and to the left, etc.}
\label{fig:square-state}
\end{figure}

A time step in the model is described by a mapping, where we write $n_{ij}^t$ to denote the state of node $(i,j)$ at $t$,
\begin{equation}\label{eqn:mapping}
n^{t+1}_{ij} = f(n^t_{ij}, n^t_{i+1,j}, n^t_{i,j+1}, n^t_{i-1,j}, n^t_{i,j-1}).
\end{equation}
In theory, this map can be written down explicitly. The node $(i,j)$ and its four nearest neighbors have a total of $2^{20}\approx 10^6$ states, and these states map to the $2^4=16$ possible states $n^{t+1}_{ij}$. In other words, the map is
$
f: \{0,1\}^{20} \mapsto \{0,1\}^4.
$
A more practical representation is divided into two operations, propagation and collision. 

In the propagation operation, each particle moves to the next node (see figure \ref{fig:propagation}). All nodes are equidistant in the lattice, which means that the particles all have the same speed.

In the collision operation, collisions between particles are considered. Two particles in the same node can collide with each other, so that they change directions. With the restriction that collisions must preserve momentum, one can verify that the only allowed collision rules are the ones shown in figure \ref{fig:collision}. All other states must map to themselves in the collision operation.

The lattice gas model can be propagated by alternating between the collision and propagation operations. Applying both operators in sequence is equivalent to taking one time step according to equation \eqref{eqn:mapping}. This concludes a basic description of the model on a square lattice.

\begin{figure}[htp]
\centering
\subfigure[]{
  \includegraphics{figs/propagation-before.pdf}
  }
\subfigure[]{
  \includegraphics{figs/propagation-after.pdf}
  }
\caption{Example of how the propagation operation works on a square lattice. To the left is the situation before propagation, and to the right after propagation.}
\label{fig:propagation}
\end{figure}

\begin{figure}[htp]
\centering
\subfigure[]{
  \includegraphics{figs/collision-before.pdf}
  }
\subfigure[]{
  \includegraphics{figs/collision-after.pdf}
  }
\caption{Example of how collision works on a square lattice. The only configuration that actually leads to a change are the ones shown in the middle and in the lower right corner.}
\label{fig:collision}
\end{figure}

\subsection{Extension to hexagonal lattice}
It is possible to construct the lattice gas model on a hexagonal lattice analogous to the square lattice model. In the hexagonal lattice, each node has six neighbors instead of four, and each node has $2^6$ possible states. On the hexagonal lattice, the collision operation is slightly more complicated, and it turns out the collision rules can be chosen in more than one way. The details of our hexagonal implementation are treated briefly in section \ref{sec:imp}.


\section{Implementation details}
\label{sec:imp}
% describe bouncing off walls, random initializations at sources... anything else?

Unlike the square lattice, the hexagonal lattice gives us several possible collision 
rules that preserve momentum. This is a consequence of the increased degree of freedom
in the system. Now, two particles can collide along three different axes.
Also, collisions between
two and four particles can be realized in more than one way.
There are two approaches to dealing with this. The first is to
choose an allowed collision rule randomly for each collision and the second is to
have a deterministic mapping scheme which always is the same. Our implementation
uses the second approach with a mapping scheme which is cyclic, meaning that by
applying a collision rule a given number of times, an arbitrary state should map back to
itself.
The mapping scheme for our implementation can be seen in table \ref{collisionscheme}. 
All remaining states map to themselves.
We wanted to make the collisions reversible, which is why we why we chose this mapping.

\begin{table}\centering
\caption{The collision scheme used for the hexagonal lattice.
Format: (nw,w,sw,se,e,ne), nw=northwest etc.}
\label{collisionscheme}
\begin{tabular}{ccc}
(0,1,0,1,0,1)& $\mapsto$ & (1,0,1,0,1,0) \\
(1,0,1,0,1,0)& $\mapsto$ & (0,1,0,1,0,1) \\
(0,1,0,0,1,1)& $\mapsto$ & (1,0,0,1,0,1) \\
(1,0,0,1,0,1)& $\mapsto$ & (0,1,0,0,1,1) \\
(1,1,0,0,1,0)& $\mapsto$ & (1,0,1,0,0,1) \\
(1,0,1,0,0,1)& $\mapsto$ & (1,1,0,0,1,0) \\
(0,1,0,1,1,0)& $\mapsto$ & (0,0,1,1,0,1) \\
(0,0,1,1,0,1)& $\mapsto$ & (0,1,0,1,1,0) \\
(0,1,1,0,1,0)& $\mapsto$ & (1,0,1,1,0,0) \\
(1,0,1,1,0,0)& $\mapsto$ & (0,1,1,0,1,0) \\
(0,1,1,0,0,1)& $\mapsto$ & (1,1,0,1,0,0) \\
(1,1,0,1,0,0)& $\mapsto$ & (0,1,1,0,0,1) \\
(1,0,0,1,1,0)& $\mapsto$ & (0,0,1,0,1,1) \\
(0,0,1,0,1,1)& $\mapsto$ & (1,0,0,1,1,0) \\
(1,0,1,1,0,1)& $\mapsto$ & (0,1,1,0,1,1) \\
(0,1,1,0,1,1)& $\mapsto$ & (1,1,0,1,1,0) \\
(1,1,0,1,1,0)& $\mapsto$ & (1,0,1,1,0,1) \\
(0,0,1,0,0,1)& $\mapsto$ & (0,1,0,0,1,0) \\
(0,1,0,0,1,0)& $\mapsto$ & (1,0,0,1,0,0) \\
(1,0,0,1,0,0)& $\mapsto$ & (0,0,1,0,0,1) \\
\end{tabular}
\end{table}


\subsection{Bouncing off walls}
Bouncing off walls can be implemented in many different ways.
To keep our implementation as simple as possible, the collision
mapping just reverses all particle directions. This corresponds to no-slip bouncing.

\subsection{Sources and sinks}
Sources have been implemented by setting the state of source nodes to be fully
filled with particles. This happens with 
a probability $p_{source}=0.8$ in each time step and otherwise nothing special
happens. This has the side effect that particles near the source may move into the
source and disappear. It turns out that for the chosen setting, the average flux
out of the source is larger than the average flux into the source and thus it really
works like a source and not a sink, unless the particle density around the source is close to one.

The implementation of a sink is simple. In each time step, the state of the sink node is set
to having no particles at all, meaning that all particles that come into the sink 
vanish. 


\section{Experiments, results}
\label{sec:exp}
% or should this just say ``waves''?
\subsection{Waves}
We start out with a square area, surrounded by walls, where initially each lattice site (not including walls)
is completely filled with a probability of 0.3. Completely filled means that it contains six or
four particles, if we use a hexagonal or a square lattice, respectively. After one time step,
this looks completely random, as can be seen in figure \ref{hexwaveinit}. In this figure, and the
next few figures, we used a hexagonal lattice. The effects of using a square lattice instead will be
explained in the last part of this section.

\begin{figure}[htp]
% if we change to just waves, this caption needs to be changed.
\centering
  \includegraphics[width=100pt]{figs/hexwaveinit.png}
\caption{Initial conditions, after one time step, for the wave experiment using a hexagonal lattice.}
\label{hexwaveinit}
\end{figure}

By clicking the left mouse button we set a circular area of lattice sites around the mouse cursor to
be completely filled. Just like a real gas, on average these added particles move towards areas of lower
concentrations of particles, away from the spawn point (since the circular area has the highest concentration
possible). Three frames taken just after the particles were added are shown in figure \ref{hexwavestart}.

\begin{figure}[htp]
\centering
  \includegraphics[width=300pt]{figs/hexwavestart.png}
\caption{Frames demonstrating the initial phase of the wave, using a hexagonal lattice.}
\label{hexwavestart}
\end{figure}


The resulting waves bounce off against the walls, figure \ref{hexwavebounce}.
\begin{figure}[htp]
\centering
  \includegraphics[width=100pt]{figs/hexwavebounce.png}
\caption{Just like real waves, the lattice gas wave bounce off the walls, using a hexagonal lattice.}
\label{hexwavebounce}
\end{figure}

Eventually, through collisions
with the other random particles in the box, the particles we added spread out and the pattern which was formed
when we filled the circular area is lost. Figure \ref{hexwaveend} shows what it looks like many time
steps later (approximately 30 seconds later, when the simulation is run on a fast computer).
We can see no trace of the wave.

\begin{figure}[htp]
\centering
  \includegraphics[width=100pt]{figs/hexwaveend.png}
\caption{Many updates later, the picture looks the same as before the wave, using a hexagonal lattice.}
\label{hexwaveend}
\end{figure}


% compare to analytical results, if possible?

If we use a square lattice instead, we still get the wave-like behaviour, figure \ref{squarewave}.
\begin{figure}[htp]
\centering
  \includegraphics[width=100pt]{figs/squarewave.png}
\caption{The wave looks similar when using a square lattice.}
\label{squarewave}
\end{figure}

However, the waves persist for much longer. One possible explanation for this could be that there are
fewer possible collisions.
Even after letting the simulation run for a long time after the particles were added, traces of waves
going back and forth can still be seen. Unfortunately, this is not clear from a static picture, so
it is not possible to show them in this report.

\subsection{Diffusion}
In this experiment, we have three boxes in a row, and the gas can spread between adjacent boxes through a small
hole in the separating wall. Initially, all boxes are empty except for the first one, where each lattice site
is completely filled with particles. The initial conditions are shown in figure \ref{diffusioninit}.

\begin{figure}[htp]
% if we change to just waves, this caption needs to be changed.
\centering
  \includegraphics[width=200pt]{figs/diffusioninit.png}
\caption{Initial conditions, after one time step, for the diffusion experiment.}
\label{diffusioninit}
\end{figure}

As the simulation runs, the gas spreads first to the second box (figure \ref{diffusionbox2fill}).
\begin{figure}[htp]
\centering
  \includegraphics[width=200pt]{figs/diffusionbox2fill.png}
\caption{Much of the gas has spread to the second box.}
\label{diffusionbox2fill}
\end{figure}

Eventually, particles also start to fill up the third box (figure \ref{diffusionbox3fill}).
\begin{figure}[htp]
\centering
  \includegraphics[width=200pt]{figs/diffusionbox3fill.png}
\caption{A while later, many particles are also found in the third box.}
\label{diffusionbox3fill}
\end{figure}

Given enough time, they spread out uniformly over the whole available space, just like a real gas would.
This ``steady-state'' is shown in figure \ref{diffusionend}.
\begin{figure}[htp]
\centering
  \includegraphics[width=200pt]{figs/diffusionend.png}
\caption{Eventually, the particle density is the same for all boxes.}
\label{diffusionend}
\end{figure}

We modified our program to count the number of particles in each box; see figure \ref{gascount}.
% compare to theoretical results?
\begin{figure}[htp]
\centering
  \includegraphics[width=300pt]{figs/gascount.pdf}
\caption{Particle density in each box as a function of the number of updates.}
\label{gascount}
\end{figure}

The results in figure \ref{gascount} can be compared qualitatively to the following model
\[
\left\{
\begin{array}{l}
  \frac{dc_1}{dt}=k(-c_1+c_2)\\
  \frac{dc_2}{dt}=k(c_1-2c_2+c_3)\\
  \frac{dc_3}{dt}=k(c_2-c_3)
\end{array}\right.
\]
Where $c_1$ denotes the particle density in box $1$, $c_2$ denotes the particle density in box $2$
, $c_3$ denotes the particle density in box $3$ and $k$ is a constant determining the speed of
the particle flows. This system of coupled differential equations can be solved analytically and
the solutions are
\begin{equation}
\label{simplediff}
\left\{
\begin{array}{l}
  c_1=\frac{1}{12}(e^{-5kt}+4+7e^{-kt})\\
  c_2=\frac{1}{3}(1-e^{-3kt})\\
  c_3=\frac{1}{3}-\frac{1}{12}(e^{-5kt}+7e^{-kt}-4e^{-3kt})
\end{array}\right.
\end{equation}
In an attempt to match the time $t$ with the previous time scale 
the $c_2$-curve was fitted to the green curve in figure \ref{gascount} which gave a value of
$k\approx 1.4032 \cdot 10^{-4}$. With this value of $k$, the analytical solutions was plotted
as a function of time. The result can be seen in figure \ref{gascount_anal}. One can see that
the behaviour here is similar to the one in figure \ref{gascount}, but the relative behaviour
of the curves is different. It is to be expected that the analytical results differ from the
simulation since the model upon which the analytical results rest is simplistic, not 
taking into account the features of the spatial geometry used in the simulation. It is
likely that the flows in the simulations are not the same between box $1$ and $2$ as between
box $2$ and $3$, which would possibly explain part of the difference between figure \ref{gascount}
and figure \ref{gascount_anal}. 
However, from the qualitative similarity one can hope that our simulation qualitatively
captures some features of diffusion driven flow.

\begin{figure}[htp]
\centering
  \includegraphics[width=300pt]{figs/gascount_anal.pdf}
\caption{Analytical solution to the simple differential equation model in eqn. \eqref{simplediff}, meant to approximate the diffusion experiment.}
\label{gascount_anal}
\end{figure}

\subsection{Flow around cylinder}
This experiment was conducted to examine what this model capture with respect to gas flow around a cylinder shaped obstacle. We have a rectangular domain, with walls at the top, bottom and left sides. Next to the wall at the left side, the entire strip is a source. At the right side of the domain, the entire strip is a sink. In this way, particles are made to enter the system at the left side and leave the system at the right side, creating a flow of particles from left to right. The domain can be thought of as a part of a pipe, with a gas flowing through it. Roughly at the middle of the domain is a round obstacle, a cylinder going straight through the pipe, perpendicular to the direction of the flow.

After a short while the domain is filled with particles and a somewhat steady state is reached. This state can be seen in figure~\ref{flow1}. As can be seen in this figure, the gas does not behave realistically. Instead a pattern of unrealistic density differences is created. The pattern is clearly influenced by the hexagonal lattice that is used.
\begin{figure}[htp]
\centering
  \includegraphics[width=300pt]{figs/scenario3mathexprintscreen1.png}
\caption{The pattern of different densities is unrealistic and artifacts coming from the hexagonal lattice are apparent.}
\label{flow1}
\end{figure}

Unexpectedly and seemingly without reason, at times the density becomes higher at a certain point to the right of the cylinder. This can be seen in figure~\ref{flow2}. A wave is created and moves towards the cylinder as seen in figures~\ref{flow3} and~\ref{flow4}. This happens repeatedly, but the times between wave formations seem to be irregular. Therefore we suspect that the system shows turbulent behaviour.
Note, however, that the source randomly spawns particles with uniform distribution. In spite of this,
the waves show a clear structure. We find this surprising.

\begin{figure}[htp]
\centering
  \includegraphics[width=300pt]{figs/scenario3mathexprintscreen10.png}
\caption{Waves form after the cylinder, without any obvious cause. The formation of waves does not seem to be periodic in time. The waves have a direction that goes against the flow, they move towards the cylinder.}
\label{flow2}
\end{figure}

\begin{figure}[htp]
\centering
  \includegraphics[width=270pt]{figs/scenario3mathexprintscreen5.png}
\caption{The wave bounces against the cylinder.}
\label{flow3}
\end{figure}

\begin{figure}[htp]
\centering
  \includegraphics[width=270pt]{figs/scenario3mathexprintscreen6.png}
\caption{The wave moves away from the cylinder and diffuses. Soon the system will be back in the ``normal'' state, seen in figure~\ref{flow1}.}
\label{flow4}
\end{figure}

\section{Conclusion}
\label{sec:conclusion}
As was argued above, lattice gas automata capture the qualitative behaviour of a gas when it evens out density differences by diffusion. They also capture the qualitative behavior of waves of density difference in a gas that fills a space, for example sound waves.

%% Then a gas flows through space from one side to the other, for example in a pipe, the gas is expected to behave in a certain way when flowing around an obstacle. Typically, the density should increase close to the obstacle on the side of the obstacle that is facing the flow, and the density should decrease on the other side of the obstacle. Also, some turbulence and strong currents are expected right after the interaction with the obstacle. The turbulent behaviour is expected to even out further away from the obstacle, making the gas behave as if it never interacted with the obstacle. Lattice gas automata does not capture this behaviour. Instead we found in experiments that the simulation became turbulent in the entire domain after the obstacle. This domain did not show an even density over different parts of the domain, and waves going against the direction of the flow formed seemingly randomly.

We conclude that this model can be useful for modelling diffusion and waves, but less useful for modelling flows. Our simulations showed that the model appears to be able to show turbulent behaviour. The main benefits of using cellular automata are that the relatively simple rules of the model make it easy to implement and, and the basic rules are easy to understand. Using software that makes use of the many cores of modern processors, it becomes possible simulate a huge number of particles (on the order of several hundred thousand) in a satisfactory pace as they are being computed.

\end{document}
